\noindent
In the following, the binaries in \commandName{red}, the options in \optionName{green}, the inputs are in  \textbf{black} and the outputs in  \outputName{blue}. 

\section{Head Matrix assembly $\mathbf{HeadMat}$:}
\label{sect: command assemble HeadMat}

\noindent
Inputs: 
\begin{itemize}
    \item \fileName{subject.geom}: geometry description file (see Appendix~\ref{sec:geom})
    \item \fileName{subject.cond}: conductivity description file (see Appendix~\ref{sec:cond})
\end{itemize}

\noindent
Output:
\begin{itemize}
    \item \outputName{HeadMat.bin}: binary file containing the matrix $\mathbf{HeadMat}$ (symmetric format).
\end{itemize}
The symmetric format only stores the lower half of a matrix.
\medskip

\noindent
\command{\commandName{om\_assemble} \optionName{-HeadMat} subject.geom subject.cond \outputName{HeadMat.bin}}
\medskip
Note: the abbreviated option names \optionName{-HM} or \optionName{-hm} can be used instead of \optionName{-HeadMat}.

\section{Source matrix assembly $\mathbf{Source}$:}
\label{sect: command assemble SourceMat}

\noindent
Inputs: 
\begin{itemize}
    \item \fileName{subject.geom}: geometry description file (see Appendix~\ref{sec:geom})
    \item \fileName{subject.cond}: conductivity description file (see Appendix~\ref{sec:cond})
    \item the source(s):
        \begin{description}
            \item [[dipolar case]] \fileName{dipolePosition.dip}: dipole description file (list of coordinates and orientations)
                                    (see Appendix~\ref{sec:dipoles}) 
            \item [[case of distributed sources]]  \fileName{sourcemesh}: source mesh (accepted formats:  *.tri or *.mesh of BrainVisa, or *.vtk) 
        \end{description}
\end{itemize}

\noindent
Output:
\begin{itemize}
    \item \outputName{SourceMat.bin}: binary file containing $\mathbf{SourceMat}$
\end{itemize}

\medskip

\noindent
For dipolar sources:\\
\noindent
\command{\commandName{om\_assemble} \optionName{-DipSourceMat} subject.geom subject.cond dipolePosition.dip \outputName{SourceMat.bin}}
\medskip
Note:  the abbreviated option names \optionName{-DSM} or \optionName{-dsm} can be used instead of \optionName{-DipSourceMat}.

\medskip

\noindent
For distributed sources:\\
\noindent
\command{\commandName{om\_assemble} \optionName{-SurfSourceMat} subject.geom subject.cond sourcemesh \outputName{SourceMat.bin}}
\medskip
Note: the abbreviated option names  \optionName{-SSM} or \optionName{-ssm}  can be used instead of  \optionName{-SurfSourceMat}.

\section{$\mathbf{HeadMat}$ matrix inversion:}
\label{sect: command invert HeadMat}

\noindent
Inputs:
\begin{itemize}
    \item \fileName{HeadMat.bin}: binary file containing matrix $\mathbf{HeadMat}$ (symmetric format)
\end{itemize}

\noindent
Output:
\begin{itemize}
    \item \outputName{HeadMatInv.bin}: binary file containing matrix $\mathbf{HeadMat}^{-1}$ (symmetric format)
\end{itemize}

\medskip

\noindent
\command{\commandName{om\_minverser} HeadMat.bin \outputName{HeadMatInv.bin}}

\section{Linear transformation from X to the sensor potential:}
\label{sect: command assemble sensors}

\checkItem \underline{For EEG}:\\

A linear interpolation is computed which relates X to the electrode potential through the linear transformation:
\[
    \mathbf{V_{electrode}} = \mathbf{Head2EEG} . \mathbf{X}
\]
where:
\begin{itemize}
    \item $\mathbf{V_{electrode}}$ is the column-vector of potential values at the sensors (output of EEG forward problem),
    \item $\mathbf{X}$ is the column-vector containing the values of the potential and the normal current on all the surface of the model,
    \item $\mathbf{Head2EEGMat}$ is the linear transformation to be computed.
\end{itemize}

\bigskip

\noindent
Inputs:
\begin{itemize}
    \item \fileName{subject.geom}: geometry description file (see Appendix~\ref{sec:geom})
    \item \fileName{subject.cond}: conductivity description file (see Appendix~\ref{sec:cond})
    \item \fileName{patchespositions.txt}: file containing the positions of the EEG electrodes (see Appendix~\ref{sec:sensors})
\end{itemize}
Sortie:
\begin{itemize}
    \item \outputName{Head2EEGMat.bin}: file containing the matrix $\mathbf{Head2EEGMat}$ (sparse format)
\end{itemize}
The sparse format allows to store efficiently matrices containing a small proportion of non-zero values.
\medskip

\noindent
\command{\commandName{om\_assemble} \optionName{-Head2EEGMat} subject.geom subject.cond patchespositions.txt \outputName{Head2EEGMat.bin}}
\medskip
Note: the abbreviated option names \optionName{-H2EM} or \optionName{-h2em} can be used instead of \optionName{-Head2EEGMat}.

\bigskip

\checkItem \underline{For MEG}:\\
In the case of MEG there are more matrices to assemble, as explained in section~\ref{}. The magnetic field is related both to the sources directly, as well as to the electric  potential, according to:
\[
    \mathbf{M_{sensor}} = \mathbf{Source2MEGMat} . \mathbf{S} + \mathbf{Head2MEGMat}.\mathbf{X}
\]

\medskip

\noindent
\underline{Contribution to MEG from the potential ($\mathbf{Head2MEGMat}$)}:\\
Inputs:
\begin{itemize}
    \item \fileName{subject.geom}: geometry description file (see Appendix~\ref{sec:geom})
    \item \fileName{subject.cond}: conductivity description file (see Appendix~\ref{sec:cond})
    \item \fileName{sensorpositions.txt}: positions and orientations of MEG sensors (see Appendix~\ref{sec:sensors})
\end{itemize}
Output:
\begin{itemize}
    \item \outputName{Head2MegMat.bin}: binary file containing $\mathbf{Head2MEGMat}$
\end{itemize}

\medskip

\noindent
\command{\commandName{om\_assemble} \optionName{-Head2MEGMat} subject.geom subject.cond sensorpositions.txt \outputName{Head2MEGMat.bin}}
\medskip
Note:  the abbreviated option names \optionName{-H2MM} or \optionName{-h2mm} can be used instead of \optionName{-Head2MEGMat}.

\bigskip

\noindent
\underline{Contribution to MEG from the sources ($\mathbf{Source2MEGMat}$)}:\\
Inputs:
\begin{itemize}
    \item the source(s):
    \begin{description}
        \item [[dipolar sources]] \fileName{dipolePosition.dip}:  dipole description file (list of coordinates and orientations) (see Appendix~\ref{sec:dipoles}) 
        \item [[distributed sources]] \fileName{sourcemesh}:  source mesh (accepted formats:  *.tri or *.mesh of BrainVisa, or *.vtk) 
    \end{description}
    \item \fileName{sensorpositions.txt}: positions and orientations of MEG sensors (see Appendix~\ref{sec:sensors})
\end{itemize}
Output: 
\begin{itemize}
    \item \outputName{DipSource2MEGMat.bin}: binary file containing $\mathbf{DipSource2MEGMat}$ \\
\item or \outputName{SurfSource2MEGMat.bin}: binary file containing $\mathbf{SurfSource2MEGMat}$ 
\end{itemize}

\medskip

\noindent
For dipolar sources:\\
\noindent
\command{\commandName{om\_assemble} \optionName{-DipSource2MEGMat} dipolePosition.dip sensorpositions.txt \outputName{DipSource2MEGMat.bin}}
\medskip
Note:  the abbreviated option names \optionName{-DS2MM} or \optionName{-ds2mm} can be used instead of \optionName{-DipSource2MEGMat}.

\medskip

\noindent
For distributed sources:\\
\noindent
\command{\commandName{om\_assemble} \optionName{-SurfSource2MEGMat} sourcemesh sensorpositions.txt \outputName{SurfSource2MEGMat.bin}}
\medskip
Note:   the abbreviated option names  \optionName{-SS2MM} or \optionName{-ss2mm} can be used instead of \optionName{-SurfSource2MEGMat}.

\section{Gain matrix computation:}
\label{sect: command gain}

The gain matrix represents the linear transformation relating the activation of sources, at \textbf{predefined positions and orientations} to the values of the fields of interest (electric potential or magnetic field) at predefined sensor positions (and orientations for MEG). 

\checkItem \underline{For EEG}:\\
Inputs:
\begin{itemize}
    \item \fileName{HeadMatInv.bin}: binary file containing $\mathbf{HeadMat}^{-1}$ (symmetric format)
    \item \fileName{SourceMat.bin}: binary file containing either  $\mathbf{SurfSourceMat}$ or  $\mathbf{DipSourceMat}$
    \item \fileName{Head2EEGMat.bin}: binary file containing $\mathbf{Head2EEGMat}$ (sparse format)
\end{itemize}
Output:
\begin{itemize}
    \item \outputName{GainEEGMat.bin}: binary file contining the gain matrix
\end{itemize}

\medskip

\noindent
\command{\commandName{om\_gain} \optionName{-EEG} HeadMatInv.bin SourceMat.bin Head2EEGMat.bin \outputName{GainEEGMat.bin}}


\bigskip

\checkItem\underline{For MEG}:\\
Inputs:
\begin{itemize}
    \item \fileName{HeadMatInv.bin}: binary file containing $\mathbf{HeadMat}^{-1}$ (symmetric format)
    \item \fileName{SourceMat.bin}: binary file containing either  $\mathbf{SurfSourceMat}$ or  $\mathbf{DipSourceMat}$
    \item \fileName{Head2MEGMat.bin}: binary file containing $\mathbf{Head2MEGMat}$
    \item \fileName{Source2MEGMat.bin}:binary file containing either  $\mathbf{DipSource2MEGMat}$ or $\mathbf{SurfSource2MEGMat}$
\end{itemize}
Output:
\begin{itemize}
    \item \outputName{GainMEGMat.bin}: binary file containing the gain matrix
\end{itemize}

\medskip

\noindent
\command{\commandName{om\_gain} \optionName{-MEG} HeadMatInv.bin SourceMat.bin Head2MEGMat.bin Source2MEGMat.bin \outputName{GainMEGMat.bin}}


\section{The forward problem:}
\label{sect: command direct}

Once the gain matrix is computed, the forward problem amounts to defining the source activation, and applying the gain matrix to this activation.

Inputs: 
\begin{itemize}
    \item \fileName{GainMat.bin}: binary file containing EEG or MEG gain matrix
    \item \fileName{activationSources.txt}: file describing the source activation (see Appendix~\ref{sec:activ})
    \item \fileName{noise}: noise (zero, or positive real number)
\end{itemize}
Output:
\begin{itemize}
    \item \outputName{sensorResults.txt}: file containing the simulated sensor data.
\end{itemize}

\medskip

\noindent
\command{\commandName{om\_forward} gainMat.bin activationSources.txt \outputName{sensorResults.txt} noise}
